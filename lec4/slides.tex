\documentclass[beamer, en, version=2.0]{huangfusl-template}
\usepackage[scheme=plain]{ctex}
\usepackage{listings}
\usepackage{libertine}
\usefonttheme[onlymath]{serif}
% Change tt font to Source Code Pro
\usepackage{sourcecodepro}
\lstset{
    basicstyle=\ttfamily\footnotesize,
    backgroundcolor=\color{darkblue!10},
    % Set color paddings
    frame=single,
    framerule=0pt,
    breakatwhitespace=true,
    showstringspaces=false,
    % Set comment to gray and italic
    commentstyle=\color{gray}\itshape,
    % Set keyword to bold and darkblue
    keywordstyle=\color{darkblue}\bfseries,
    % Set string to darkred
    stringstyle=\color{darkred},
}

\title[Lecture 4: Data visualization]{\LARGE{清华经管博士生工作坊}\newline\newline \large{Lecture 4: Data visualization}}
\author{皇甫硕龙}
\date{2024-04-25}
\splitsections

\begin{document}
    \begin{frame}
        \maketitle
    \end{frame}

    \section{Use Pandas for Data Pre-processing}

    \begin{frame}
        \frametitle{Use Pandas for Data Pre-processing}

        The procedures in data pre-processing include:

        \begin{enumerate}
            \item Data I/O, which has been covered in the previous lecture.
            \item Selection: extract certain rows or columns from the dataset.
            \item Transformation \& Aggregation: apply functions on the dataset.
            \item Merge: combine multiple tables into one.
        \end{enumerate}
    \end{frame}

    \begin{frame}[fragile]
        \frametitle{Starting DataFrames}

        For the following slides, we will use the following DataFrame as an example:

\begin{lstlisting}[language=python]
import pandas as pd
import numpy as np
np.random.seed(0)

dates = pd.date_range("20130101", periods=6)
df = pd.DataFrame(np.random.randn(6, 4), index=dates, columns=list("ABCD"))

#                    A         B         C         D
# 2013-01-01  1.764052  0.400157  0.978738  2.240893
# 2013-01-02  1.867558 -0.977278  0.950088 -0.151357
# ... 4 more rows ...
\end{lstlisting}

    \end{frame}

    \subsection{Selection}

    \begin{frame}[fragile]
        \frametitle{Selection: Using GetItem ({\normalsize\texttt{[]}})}

        \begin{itemize}
            \item Providing a column name yields the column as a Series.
            \item Providing a list of column names yields the columns as a DataFrame.
            \item Providing a slice of row indexes yields the rows as a DataFrame.
        \end{itemize}

\begin{lstlisting}[language=python]
df['A'] # Get column A, [1.764052, 1.867558, ...]
df[['A', 'B']] # Get columns A and B, [[1.764052, 0.400157], ...]
df[0:3]
# Get the first 3 rows, [[1.764052, 0.400157, 0.978738, 2.240893], ...]
\end{lstlisting}

    \end{frame}

    \begin{frame}[fragile]
        \frametitle{Selection: Using Label-based Indexing}

        \begin{itemize}
            \item Use {\color{blue}\footnotesize\verb|loc|} to select data by [row, column] labels.
            \item Use {\color{blue}\footnotesize\verb|iloc|} to select data by [row, column] numerical index (like acting on a NumPy array).
        \end{itemize}

\begin{lstlisting}[language=python]
df.loc[:, ['A', 'B']] # Get columns A and B
df.loc['20130102':'20130104', ['A', 'B']] # Get rows 2-4 and columns A and B
df.iloc[3] # Get the 4th row
df.iloc[3:5, 0:2] # Get rows 4-5 and columns 1-2
\end{lstlisting}
    \end{frame}

    \begin{frame}[fragile]
        \frametitle{Selection: Boolean Indexing}

        We can use boolean indexing to select data based on certain criteria. The provided boolean array can be a Series or a DataFrame.

        \begin{itemize}
            \item \textbf{Series}: Select rows where the value is {\footnotesize\verb|True|}.
            \item \textbf{DataFrame}: Elements with {\footnotesize\verb|False|} values will be replaced by {\footnotesize\verb|NaN|}.
        \end{itemize}

\begin{lstlisting}[language=python]
df[df['A'] > 0] # Select rows where A > 0
#                    A         B         C         D
# 2013-01-01  1.764052  0.400157  0.978738  2.240893
# 2013-01-02  1.867558 -0.977278  0.950088 -0.151357
# 2013-01-04  0.761038  0.121675  0.443863  0.333674
# 2013-01-05  1.494079 -0.205158  0.313068 -0.854096
df[df > 0] # Replace elements with A <= 0 by NaN
\end{lstlisting}
    \end{frame}
    \begin{frame}[fragile]
        \frametitle{Selection: Setting}

        Selections can also be used to set values in the DataFrame. Only the selected elements will be modified.

\begin{lstlisting}[language=python]
df2 = df.copy()
df2[df2 > 0] = -df2 # Replace elements with A > 0 by -A
#                    A         B         C         D
# 2013-01-01 -1.764052 -0.400157 -0.978738 -2.240893
# 2013-01-02 -1.867558 -0.977278 -0.950088 -0.151357
# ... 4 more rows ...
\end{lstlisting}

        Similar to {\color{blue}\footnotesize\verb|loc|} and {\color{blue}\footnotesize\verb|iloc|}, we can use {\color{blue}\footnotesize\verb|at|} and {\color{blue}\footnotesize\verb|iat|} to set values at specific locations.
    \end{frame}
    \begin{frame}[fragile]
        \frametitle{Selection Special: Handling Missing Data}

        Missing data are {\footnotesize\verb|NaN|}s in Pandas.

        \begin{itemize}
            \item Use {\color{blue}\footnotesize\verb|dropna()|} to drop rows with missing data.
            \item Use {\color{blue}\footnotesize\verb|fillna()|} to fill missing data with a certain value.
            \item We can use {\color{blue}\footnotesize\verb|isna()|} and {\color{blue}\footnotesize\verb|notna()|} to check whether the data is missing. The return value is a boolean DataFrame.
        \end{itemize}
    \end{frame}
    \subsection{Transformation \& Aggregation}
    \begin{frame}[fragile]
        \frametitle{Aggregation: Stats}

        Pandas provides a variety of functions to calculate statistics on the DataFrame. Without specifying the {\color{blue}\footnotesize\verb|axis|} parameter, the function will be applied to the entire DataFrame, otherwise it will be applied to the specified axis. The functions will \textbf{ignore} missing data by default.

        \begin{itemize}
            \item These functions have been covered in the previous lecture.
            \item {\color{blue}\footnotesize\verb|value_counts|}: Count the number of unique values.
        \end{itemize}
    \end{frame}
    \begin{frame}[fragile]
        \frametitle{Aggregation: Custom Combinations}

        We can use {\color{blue}\footnotesize\verb|agg()|} to perform custom aggregation functions on the DataFrame.

\begin{lstlisting}[language=python]
df.agg(['sum', 'mean']) # Calculate the sum and mean of each column
#              A         B         C         D
# sum   3.230518  0.403613  3.694237  2.281223
# mean  0.538420  0.067269  0.615706  0.380204
df.agg({'A': 'mean', 'B': 'sum'}) # Calculate the mean of A and sum of B
# A   0.538420
# B   0.403613
# dtype: float64
\end{lstlisting}
    \end{frame}
    \begin{frame}[fragile]
        \frametitle{Aggregation: Custom Functions}

        {\color{blue}\footnotesize\verb|agg()|} also accepts custom functions as it's argument. The function should take a Series as input and return a scalar. Use {\color{blue}\footnotesize\verb|axis|} parameter to specify the aggregated axis.

\begin{lstlisting}[language=python]
def double_mean(x):
    return 2 * x.mean()
df.agg(double_mean) # Axis defaults to 0
# A    1.076839
# B    0.134538
# C    1.231412
# D    0.760408
# dtype: float64
\end{lstlisting}
    \end{frame}
    \begin{frame}[fragile]
        \frametitle{Transformation: Custom Functions}

        We can use {\color{blue}\footnotesize\verb|transform()|} to apply custom functions to the DataFrame. Transformation are performed column-wise or row-wise, depending on the {\color{blue}\footnotesize\verb|axis|} parameter. Thus, the function provided should accept a Series as input and return a Series of the same length.

\begin{lstlisting}[language=python]
df.transform(lambda x: x - x.mean()) # Subtract the mean of each column
#                    A         B         C         D
# 2013-01-01  1.225633  0.332888  0.363032  1.860689
# 2013-01-02  1.329138 -1.044547  0.334382 -0.531561
# ... 4 more rows ...
\end{lstlisting}
    \end{frame}
    \subsection{Merge}
    \begin{frame}[fragile]
        \frametitle{Merge: Concatenation}

        Concatenation operation {\color{blue}\footnotesize\verb|concat()|} is similar to the corresponding operation in NumPy. We can concatenate multiple DataFrames along the column or row.
    \end{frame}
    \begin{frame}[fragile]
        \frametitle{Merge: Join}

        Join operation {\color{blue}\footnotesize\verb|left.join(right)|} on a column {\color{red}\footnotesize\verb|a|} is equivalent to the following actions.

        \begin{enumerate}
            \item The column {\color{red}\footnotesize\verb|a|} of the left and the index of right should have intersection.
            \item For each row in the left DataFrame, find the corresponding row in the right DataFrame by the value of column {\color{red}\footnotesize\verb|a|}.
            \item Merge the two rows into one.
            \item Perform the above steps for all rows in the left DataFrame.
        \end{enumerate}
    \end{frame}
    \section{Introduction to Data Visualization}
    \begin{frame}[fragile]
        \frametitle{Introduction to Matplotlib}

        Matplotlib is a plotting library for Python.

        \begin{itemize}
            \item Install matplotlib by running {\color{darkred}\footnotesize\verb|pip install matplotlib|}.
            \item Conventionally, we use {\color{blue}\footnotesize\verb|import matplotlib.pyplot as plt|} or {\color{blue}\footnotesize\verb|from matplotlib import pyplot as plt|} to import the library.
        \end{itemize}
    \end{frame}
    \subsection{Basic Plotting}
    \begin{frame}[fragile]
        \frametitle{Plot: Line Plot}

        \begin{columns}
        \begin{column}{0.5\textwidth}
\begin{lstlisting}[language=python, breaklines]
plt.plot([1, 2, 3, 4], [1, 4, 2, 3])
plt.plot([1, 2, 3, 4], [2, 3, 4, 5], color='red', legend='Line 2')
plt.xlabel('x')
plt.ylabel('y')
plt.title('Simple line plot')
plt.show()
\end{lstlisting}
        \end{column}
        \begin{column}{0.5\textwidth}
            \input{imgs/1-line-plot.pgf}
        \end{column}
        \end{columns}
    \end{frame}
    \begin{frame}[fragile]
        \frametitle{Plot: Scatter Plot}

        \begin{columns}
        \begin{column}{0.5\textwidth}
\begin{lstlisting}[language=python, breaklines]
x, y = np.random.rand(2, 100)
plt.scatter(x, y, s=3, color='blue', alpha=0.5)
plt.xlabel('x')
plt.ylabel('y')
plt.title('Simple scatter plot')
plt.show()
\end{lstlisting}
        \end{column}
        \begin{column}{0.5\textwidth}
            \input{imgs/2-scatter-plot.pgf}
        \end{column}
        \end{columns}
    \end{frame}
    \begin{frame}[fragile]
        \frametitle{Hist: Histogram}

        \begin{columns}
        \begin{column}{0.5\textwidth}
\begin{lstlisting}[language=python, breaklines]
data = np.random.randn(1000)
plt.hist(data, bins=30, color='green', alpha=0.5)
plt.xlabel('x')
plt.ylabel('Frequency')
plt.title('Simple histogram')
plt.show()
\end{lstlisting}
        \end{column}
        \begin{column}{0.5\textwidth}
            \input{imgs/3-histogram.pgf}
        \end{column}
        \end{columns}
    \end{frame}
    \begin{frame}[fragile]
        \frametitle{Bar: Bar Plot}

        \begin{columns}
        \begin{column}{0.5\textwidth}
\begin{lstlisting}[language=python, breaklines]
x = ['A', 'B', 'C', 'D']
y = [1, 2, 3, 4]
plt.bar(x, y, color='orange', alpha=0.5)
plt.xlabel('x')
plt.ylabel('y')
plt.title('Simple bar plot')
plt.show()
\end{lstlisting}
        \end{column}
        \begin{column}{0.5\textwidth}
            \input{imgs/4-bar-plot.pgf}
        \end{column}
        \end{columns}
    \end{frame}
    \begin{frame}[fragile]
        \frametitle{Saving Plots}

        We can save the plots using the {\color{blue}\footnotesize\verb|savefig()|} function. Matplotlib supports a variety of formats, including PNG, PDF, and SVG.
    \end{frame}
    \subsection{Fine-tuning plots}
    \begin{frame}[fragile]
        \frametitle{Error Bar}

        \begin{columns}
        \begin{column}{0.5\textwidth}
\begin{lstlisting}[language=python, breaklines]
x = ['A', 'B', 'C', 'D']
y = [1, 2, 3, 4]
plt.bar(x, y, color='orange', alpha=0.5)
plt.errorbar(x, y, yerr=0.1, fmt='o', color='red')
plt.xlabel('x')
plt.ylabel('y')
plt.title('Simple bar plot with error bars')
plt.show()
\end{lstlisting}
        \end{column}
        \begin{column}{0.5\textwidth}
            \input{imgs/5-bar-plot-error.pgf}
        \end{column}
        \end{columns}
    \end{frame}
    \begin{frame}[fragile]
        \frametitle{Markers}

        \begin{columns}
        \begin{column}{0.5\textwidth}
\begin{lstlisting}[language=python, breaklines]
x = np.linspace(0, 10, 100)
plt.plot(x, np.sin(x), marker='x', color='blue', alpha=0.5, markersize=5, markevery=10)
plt.plot(x, np.cos(x), marker='o', color='red', alpha=0.5, markersize=5, markevery=10)
plt.xlabel('x')
plt.ylabel('y')
plt.title('Markers')
plt.show()
\end{lstlisting}
        \end{column}
        \begin{column}{0.5\textwidth}
            \input{imgs/6-line-plot-marker.pgf}
        \end{column}
        \end{columns}
    \end{frame}
    \begin{frame}[fragile]
        \frametitle{Legend}

        \begin{columns}
        \begin{column}{0.5\textwidth}
\begin{lstlisting}[language=python, breaklines]
x = np.linspace(0, 10, 100)
plt.plot(x, np.sin(x), marker='x', color='blue', alpha=0.5, markersize=5, markevery=10, label='$\\sin(x)$')
plt.plot(x, np.cos(x), marker='o', color='red', alpha=0.5, markersize=5, markevery=10, label='$\\cos(x)$')
plt.xlabel('x')
plt.ylabel('y')
plt.title('Legends')
plt.legend(loc='upper right')
plt.show()
\end{lstlisting}
        \end{column}
        \begin{column}{0.5\textwidth}
            \input{imgs/7-legend.pgf}
        \end{column}
        \end{columns}

    \end{frame}
    \begin{frame}[fragile]
        \frametitle{Axis Ticks and Labels}

        \begin{columns}
        \begin{column}{0.5\textwidth}
\begin{lstlisting}[language=python, breaklines]
x = np.linspace(0, 10, 100)
plt.plot(x, np.sin(x), marker='x', color='blue', alpha=0.5, markersize=5, markevery=10)
plt.xlabel('x')
plt.ylabel('y')
plt.title('Axis Ticks and Labels')
plt.xticks(np.arange(0, 11, 2))
plt.yticks(np.arange(-1, 1.1, 0.2))
plt.show()
\end{lstlisting}
        \end{column}
        \begin{column}{0.5\textwidth}
            \input{imgs/8-axis-ticks.pgf}
        \end{column}
        \end{columns}
    \end{frame}
    \begin{frame}[fragile]
        \frametitle{Minor Ticks}

        \begin{columns}
        \begin{column}{0.5\textwidth}
\begin{lstlisting}[language=python, breaklines]
x = np.linspace(0, 10, 100)
plt.plot(x, np.sin(x), marker='x', color='blue', alpha=0.5, markersize=5, markevery=10)
plt.xlabel('x')
plt.ylabel('y')
plt.title('Minor Ticks')
plt.minorticks_on()
plt.xticks(np.arange(0, 11, 1), minor=True)
plt.xticks(np.arange(0, 11, 2), minor=False)
plt.yticks(np.arange(-1, 1.1, 0.1), minor=True)
plt.yticks(np.arange(-1, 1.1, 0.5), minor=False)
\end{lstlisting}
        \end{column}
        \begin{column}{0.5\textwidth}
            \input{imgs/9-minor-ticks.pgf}
        \end{column}
        \end{columns}
    \end{frame}
    \begin{frame}[fragile]
        \frametitle{Gridlines}

        \begin{columns}
        \begin{column}{0.5\textwidth}
\begin{lstlisting}[language=python, breaklines]
x = np.linspace(0, 10, 100)
plt.plot(x, np.sin(x), marker='x', color='blue', alpha=0.5, markersize=5, markevery=10)
plt.xlabel('x')
plt.ylabel('y')
plt.title('Gridlines')
plt.grid(True)
# Parameter `which` can be 'major', 'minor', or 'both'
# Parameter `axis` can be 'x', 'y', or 'both'
plt.show()
\end{lstlisting}
        \end{column}
        \begin{column}{0.5\textwidth}
            \input{imgs/10-gridlines.pgf}
        \end{column}
        \end{columns}
    \end{frame}
    \begin{frame}[fragile]
        \frametitle{Error Range}

        \begin{columns}
        \begin{column}{0.5\textwidth}
\begin{lstlisting}[language=python, breaklines]
x = np.linspace(0, 10, 100)
y = 2 * x
y_pos_err = 3 + 0.5 * np.random.rand(100)
y_neg_err = 2 - 0.2 * np.random.rand(100)
plt.plot(x, y, color='black', alpha=0.5)
plt.fill_between(x, y + y_pos_err, y, color='red', alpha=0.5)
plt.fill_between(x, y, y - y_neg_err, color='green', alpha=0.5)
plt.xlabel('x')
plt.ylabel('y')
plt.title('Error Range')
\end{lstlisting}
        \end{column}
        \begin{column}{0.5\textwidth}
            \input{imgs/11-error-range.pgf}
        \end{column}
        \end{columns}
    \end{frame}
    \begin{frame}[fragile]
        \frametitle{Color Maps}

        \begin{columns}
        \begin{column}{0.5\textwidth}
\begin{lstlisting}[language=python, breaklines]
x = np.linspace(0, 10, 100)
y = np.sin(x)
colors = np.linspace(0, 1, 100)
plt.scatter(x, y, c=colors, cmap='viridis')
plt.xlabel('x')
plt.ylabel('y')
plt.title('Color Maps')
plt.colorbar()
plt.show()
\end{lstlisting}
        \end{column}
        \begin{column}{0.5\textwidth}
            \import{imgs}{12-color-maps.pgf}
        \end{column}
        \end{columns}
    \end{frame}
    \begin{frame}[fragile]
        \frametitle{Image Plot}

        \begin{columns}
        \begin{column}{0.5\textwidth}
\begin{lstlisting}[language=python, breaklines]
heatmap = np.random.rand(10, 10)
plt.imshow(heatmap, cmap='viridis')
plt.colorbar()
plt.title('Image Plot')
plt.show()
\end{lstlisting}
        \end{column}
        \begin{column}{0.5\textwidth}
            \import{imgs}{13-image-plot.pgf}
        \end{column}
        \end{columns}
    \end{frame}
    \begin{frame}[fragile]
        \frametitle{Contour Plot}

        \begin{columns}
        \begin{column}{0.5\textwidth}
\begin{lstlisting}[language=python, breaklines]
x = np.linspace(-2, 2, 100)
y = np.linspace(-2, 2, 100)
X, Y = np.meshgrid(x, y)
Z = np.exp(-X**2 - Y**2)
CS = plt.contour(X, Y, Z, cmap='viridis')
plt.clabel(CS)
plt.title('Contour Plot')
plt.show()
\end{lstlisting}
        \end{column}
        \begin{column}{0.5\textwidth}
            \import{imgs}{14-contour-plot.pgf}
        \end{column}
        \end{columns}
    \end{frame}
    \begin{frame}[fragile]
        \frametitle{Filled Contour Plot}

        \begin{columns}
        \begin{column}{0.5\textwidth}
\begin{lstlisting}[language=python, breaklines]
x = np.linspace(-2, 2, 100)
y = np.linspace(-2, 2, 100)
X, Y = np.meshgrid(x, y)
Z = np.exp(-X**2 - Y**2)
plt.contourf(X, Y, Z, cmap='viridis')
plt.colorbar()
plt.title('Filled Contour Plot')
plt.show()
\end{lstlisting}
        \end{column}
        \begin{column}{0.5\textwidth}
            \import{imgs}{15-contourf-plot.pgf}
        \end{column}
        \end{columns}
    \end{frame}
    \subsection{Subplot Layout}
    \begin{frame}[fragile]
        \frametitle{Subplots}

        \begin{columns}
        \begin{column}{0.5\textwidth}
\begin{lstlisting}[language=python, breaklines]
x = np.linspace(0, 10, 100)
y1, y2 = np.sin(x), np.cos(x)
fig, (a, b) = plt.subplots(2, 1, figsize=(2.8, 2.5), layout='constrained', sharex=True)
a.plot(x, y1, color='blue', alpha=0.5)
a.set_ylabel('y')
a.set_title('sin(x)')
b.plot(x, y2, color='red', alpha=0.5)
b.set_xlabel('x')
b.set_ylabel('y')
b.set_title('cos(x)')
fig.suptitle('Subplots')
\end{lstlisting}
        \end{column}
        \begin{column}{0.5\textwidth}
            \import{imgs}{16-subplots.pgf}
        \end{column}
        \end{columns}
    \end{frame}
    \begin{frame}[fragile]
        \frametitle{Mosaic Subplot}

        \begin{columns}
        \begin{column}{0.5\textwidth}
\begin{lstlisting}[language=python, breaklines]
x = np.linspace(0, 10, 100)
y1, y2 = np.sin(x), np.cos(x)
y3 = y1 / y2
fig, axes = plt.subplot_mosaic([['a', 'b'], ['c', 'c']], figsize=(2.8, 2.5))
axes['a'].plot(x, y1, color='blue', alpha=0.5)
axes['b'].plot(x, y2, color='red', alpha=0.5)
axes['c'].plot(x, y3, color='green', alpha=0.5)
fig.suptitle('Mosaic Subplot')
\end{lstlisting}
        \end{column}
        \begin{column}{0.5\textwidth}
            \import{imgs}{17-mosaic-subplot.pgf}
        \end{column}
        \end{columns}
    \end{frame}
    \subsection{Examples}
    \begin{frame}[fragile]
        \frametitle{Example 1: Regression Result}

        Consider a linear regression model $y = 2x + 1 + \epsilon$, where $\epsilon \sim \mathcal{N}(0, 1)$. First we generate the data and fit the model.

\begin{lstlisting}[language=python]
import numpy as np
np.random.seed(0)
x = np.stack([np.ones(25), np.random.randn(25)])
beta_true = np.array([1, 2])
y = beta_true @ x + np.random.randn(25)
beta_hat = np.linalg.inv(x @ x.T) @ x @ y
# [1.07515309, 2.11469843]
\end{lstlisting}

    \end{frame}
    \begin{frame}[fragile]
        \frametitle{Example 1: Regression Result}

        \begin{columns}
        \begin{column}{0.5\textwidth}
\begin{lstlisting}[language=python, breaklines]
plt.scatter(x[1], y, color='blue', alpha=0.5, s=7)
plt.plot(x[1], beta_hat @ x, color='red', alpha=0.7)
for i in range(25):
    plt.plot([x[1, i], x[1, i]], [y[i], beta_hat @ x[:, i]], color='gray', alpha=0.5)
plt.xlabel('x')
plt.ylabel('y')
plt.title('Regression Result')
plt.show()
\end{lstlisting}
        \end{column}
        \begin{column}{0.5\textwidth}
            \input{imgs/e-1-regression-result.pgf}
        \end{column}
        \end{columns}
    \end{frame}
    \begin{frame}[fragile]
        \frametitle{Example 2: Plot Error}

        \begin{columns}
        \begin{column}{0.5\textwidth}
\begin{lstlisting}[language=python, breaklines]
plt.plot(x[1], beta_hat @ x, color='red', alpha=0.7)
x2 = x.copy()
x2.sort(axis=1)
y_err = x2[1].std() * np.sqrt(1 / x2.shape[1] + (x2[1] - x2[1].mean()) ** 2 / np.sum((x2[1] - x2[1].mean()) ** 2))
plt.fill_between(x2[1], (beta_hat @ x2) - y_err, (beta_hat @ x2) + y_err, color='red', alpha=0.3, interpolate=True)
plt.scatter(x[1], y, color='blue', alpha=0.5, s=7)
\end{lstlisting}
        \end{column}
        \begin{column}{0.5\textwidth}
            \input{imgs/e-2-regression-error.pgf}
        \end{column}
        \end{columns}
    \end{frame}
    \begin{frame}[fragile]
        \frametitle{Example 3: QQ Plot}

        \begin{columns}
        \begin{column}{0.5\textwidth}
\begin{lstlisting}[language=python, breaklines]
from scipy.stats import norm
residuals = y - beta_hat @ x
norm_res = (residuals - residuals.mean()) / residuals.std()
norm_q = norm.ppf(np.linspace(0, 1, 25))
plt.scatter(norm_q, np.sort(norm_res), color='blue', alpha=0.5)
plt.plot(norm_q, norm_q, color='red', alpha=0.7)
plt.xlabel('Theoretical quantiles')
plt.ylabel('Sample quantiles')
plt.title('QQ Plot')
\end{lstlisting}
        \end{column}
        \begin{column}{0.5\textwidth}
            \input{imgs/e-3-qq-plot.pgf}
        \end{column}
        \end{columns}
    \end{frame}
    \begin{frame}
        \mythanks
    \end{frame}
\end{document}
